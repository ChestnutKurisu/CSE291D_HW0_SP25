\documentclass[letterpaper,12pt]{article}

%%% ==================================================================
%%% Please update the following fields with your own information:
%%% ==================================================================
\newcommand{\myFirstName}{Param}
\newcommand{\myLastName}{Somane}

\def\homeworknum{0}
\def\ucsdpid{A69033076}
\def\coursename{CSE 291D}
\def\quarter{SP25 Quarter}
\def\duedate{April 1, 2025}
%%% ==================================================================

\usepackage{xcolor}
\definecolor{darkgreen}{rgb}{0,0.4,0}
\usepackage{geometry}
\usepackage{fancyhdr}
\usepackage{amsmath,amsthm,amssymb,bm}
\usepackage{graphicx}
\usepackage{pdfpages}
\usepackage{ragged2e}
\usepackage{enumitem}
\usepackage{lastpage}
\usepackage{hyperref}
\usepackage{listings}
\usepackage{booktabs}

% Adjust page margins
\geometry{margin=1in}
\setlength{\headheight}{14.5pt}
\addtolength{\topmargin}{-2.5pt}

% Set up fancyhdr
\pagestyle{fancy}
\fancyhf{}
\lhead{{\bf \coursename \ \quarter}}
\chead{{\bf Homework \homeworknum}}
\rhead{{\bf UCSD PID: \ucsdpid}}
\cfoot{\thepage}

\newcounter{problemid}
\renewcommand{\theproblemid}{\arabic{problemid}}
\newcommand{\newproblem}{\stepcounter{problemid} \vspace*{0.4cm} {\bf Problem \theproblemid} \par}

%%% --------------------------------------------------------------------
%%% Footer on last page (if needed)
%%% --------------------------------------------------------------------
\fancypagestyle{ack_footer}{
    \fancyhf{}
    \renewcommand{\headrulewidth}{0pt}
    \renewcommand{\footrulewidth}{0pt}
    \cfoot{
      \scriptsize
      \begin{minipage}[t]{0.95\textwidth}
      \textbf{Acknowledgments:} OpenAI's ChatGPT (model gpt-4o) was utilized to assist in adding comments and docstrings to Python functions, explain code implemented by me and ensure its correctness and alignment with equations of motion, and help debug code errors. The outputs from this AI model were modified with major changes. I actively reviewed, tested, and adjusted the generated code and explanations to reflect my own understanding.
      \end{minipage}
    }
}

%% Additional colors and listings setup from the original:
\definecolor{lightgray}{gray}{0.95}
\definecolor{darkblue}{rgb}{0.0,0.0,0.5}

\lstset{
    backgroundcolor=\color{lightgray},
    basicstyle=\small\ttfamily,
    keywordstyle=\color{blue}\bfseries,
    commentstyle=\itshape\color{green!40!black},
    stringstyle=\color{red},
    columns=fullflexible,
    breaklines=true,
    frame=single,
    numbers=left,
    numbersep=5pt,
    showstringspaces=false,
    tabsize=4,
    captionpos=b,
    morekeywords={vector,rate,canvas,sphere,cylinder,distant_light,local_light,box,make_trail,trail_radius,trail_color,opacity}
}

\begin{document}

\begin{center}
  {\color{darkgreen}\Large{\scshape \myFirstName \ \myLastName}} \\
  {\bf Homework \homeworknum\ - \duedate}
\end{center}

\vspace*{0.3cm}

\noindent
\textbf{Instructor:} Albert Chern \\
\textbf{Homework \homeworknum: A Chaotic Double Pendulum Simulation and Generalized N-Pendulum in Python and VPython}\\
\rule{\textwidth}{0.4pt}

\begin{enumerate}[leftmargin=*, itemsep=1em]

% ----------------------------------------------------------------------------------
\item \textbf{Overview and Motivation}

This document presents two pendulum simulation scripts implemented in Python with VPython for 3D visualization:

\begin{enumerate}
    \item A classic \emph{double pendulum} simulation (\texttt{double\_pendulum.py}).
    \item A generalized \emph{N-pendulum} simulation (\texttt{N\_pendulum.py}).
\end{enumerate}

We outline the physical equations of motion, the numerical methods used (fourth-order Runge–Kutta), and the code structure. A pastel color palette is introduced for an aesthetically pleasing visualization. The results demonstrate chaotic behavior characteristic of multi-link pendulums, and the system serves as a visual tool for educational and research purposes in advanced dynamics.

% ----------------------------------------------------------------------------------
\item \textbf{Introduction}

\textbf{Double Pendulum.} The double pendulum is a well-known physical system that exhibits chaotic behavior for certain initial conditions. It consists of two rods and two masses, with the second mass hanging from the first. Despite its deceptively simple construction, the double pendulum can display highly sensitive dependence on initial conditions, making it a quintessential example of chaos in classical mechanics.

\textbf{N-Pendulum.} More generally, one can consider a chain of $N$ masses and $N$ rods (or $N$ segments) pivoting freely in a plane. Formulating the equations of motion for $N$ bobs is significantly more involved than for the single or double pendulum. In recent work, \textit{Yesilyurt}~\cite{yesilyurt2020equationsmotionformulationpendulum} presents a derivation of these equations using both Lagrange mechanics (with an inductive approach) and a direct vector method. This approach yields a concise summation form for the $N$-pendulum’s equations of motion, which we implement in \texttt{N\_pendulum.py} using a matrix-based solver at each time step.

The primary objectives of these simulations are:
\begin{itemize}
    \item To demonstrate chaotic motion via real-time 3D rendering for both double and $N$-pendulum cases.
    \item To provide straightforward Python programs that can be easily modified for educational or experimental purposes.
    \item To showcase a pastel color scheme that softens the visual appearance of pendulum demonstrations.
\end{itemize}

% ----------------------------------------------------------------------------------
\item \textbf{Equations of Motion for the Double Pendulum}
\label{sec:equations}

Denote for the double pendulum:
\begin{itemize}
    \item $\theta_1(t)$: Angle of the first (upper) pendulum from the vertical.
    \item $\theta_2(t)$: Angle of the second (lower) pendulum from the vertical.
    \item $\omega_1 = \dot{\theta}_1$, $\omega_2 = \dot{\theta}_2$: Angular velocities.
    \item $m_1, m_2$: Masses of the two bobs.
    \item $L_1, L_2$: Lengths of the two rods.
    \item $g$: Gravitational acceleration.
\end{itemize}

The classical equations for a planar double pendulum in a gravitational field are given in
Figure~\ref{fig:double-pendulum-eqns} below.

\begin{figure*}[ht]
\centering
\[
\begin{aligned}
\dot{\theta}_1 &= \omega_1, \quad \dot{\theta}_2 = \omega_2, \\
\dot{\omega}_1 &= \frac{-g(2m_1 + m_2)\sin\theta_1 - m_2 g\sin(\theta_1 - 2\theta_2)
- 2m_2\sin(\theta_1 - \theta_2)\bigl(\omega_2^2 L_2 + \omega_1^2 L_1 \cos(\theta_1 - \theta_2)\bigr)}
{L_1 \bigl(2m_1 + m_2 - m_2 \cos(2\theta_1 - 2\theta_2)\bigr)}, \\
\dot{\omega}_2 &= \frac{2 \sin(\theta_1 - \theta_2)\Bigl(\omega_1^2 L_1 (m_1 + m_2)
+ g (m_1 + m_2)\cos\theta_1
+ \omega_2^2 L_2 \, m_2 \cos(\theta_1 - \theta_2)\Bigr)}
{L_2 \bigl(2m_1 + m_2 - m_2 \cos(2\theta_1 - 2\theta_2)\bigr)}.
\end{aligned}
\]
\caption{Equations of motion for a planar double pendulum.}
\label{fig:double-pendulum-eqns}
\end{figure*}

% ----------------------------------------------------------------------------------
\item \textbf{Equations of Motion for the N-Pendulum}

The $N$-pendulum extends the system to $N$ bobs and rods. Following \textit{Yesilyurt}~\cite{yesilyurt2020equationsmotionformulationpendulum}, one obtains a coupled set of $N$ equations:
\[
\sum_{k=1}^{N}
\left(
g\,l_j\,\sin(\theta_j)\,m_k \,\sigma_{j,k}
\;+\;m_k \,l_j^2\,\ddot{\theta_j}\,\sigma_{j,k}
\;+\;\left(\!\!\sum_{q\ge k}^{N} m_q \,\sigma_{j,q}\right)\,l_j\,l_k\,
\sin(\theta_j-\theta_k)\,\dot{\theta_j}\,\dot{\theta_k}
\;+\;\ldots
\right)
\;=\;0,
\]
where $\theta_j$ is the angle of the $j$-th bob from the vertical, $m_j$ and $l_j$ are the mass and rod length of the $j$-th pendulum link, $g$ is gravitational acceleration, and the indicator functions $\sigma_{j,k}$ and $\phi_{j,k}$ appear to handle the sums in a systematic way. In our code, we translate these summations into a matrix--vector system:
\[
\mathbf{M}(\boldsymbol{\theta},\dot{\boldsymbol{\theta}})\;\ddot{\boldsymbol{\theta}}
\;=\;-\,\mathbf{f}(\boldsymbol{\theta},\dot{\boldsymbol{\theta}}),
\]
and solve for $\ddot{\boldsymbol{\theta}}$ at each time step (see \texttt{N\_pendulum.py}).

% ----------------------------------------------------------------------------------
\item \textbf{Numerical Integration: Fourth-Order Runge–Kutta}

We employ the fourth-order Runge–Kutta (RK4) method to integrate the system in small time steps $\Delta t$:
\[
\begin{aligned}
\mathbf{k}_1 &= \mathbf{f}(\mathbf{y}_n, t_n), \\
\mathbf{k}_2 &= \mathbf{f}(\mathbf{y}_n + \tfrac{1}{2}\Delta t\,\mathbf{k}_1, t_n + \tfrac{1}{2}\Delta t), \\
\mathbf{k}_3 &= \mathbf{f}(\mathbf{y}_n + \tfrac{1}{2}\Delta t\,\mathbf{k}_2, t_n + \tfrac{1}{2}\Delta t), \\
\mathbf{k}_4 &= \mathbf{f}(\mathbf{y}_n + \Delta t\,\mathbf{k}_3, t_n + \Delta t), \\
\mathbf{y}_{n+1} &= \mathbf{y}_n + \frac{\Delta t}{6}(\mathbf{k}_1 + 2\mathbf{k}_2 + 2\mathbf{k}_3 + \mathbf{k}_4),
\end{aligned}
\]
where for the double pendulum $\mathbf{y} = [\theta_1, \omega_1, \theta_2, \omega_2]$, and for the $N$-pendulum, $\mathbf{y} = [\theta_0,\dots,\theta_{N-1},\,\omega_0,\dots,\omega_{N-1}]$. The function $\mathbf{f}$ encapsulates the appropriate equations of motion.

% ----------------------------------------------------------------------------------
\item \textbf{Implementation in Python and VPython}
\label{sec:implementation}

\textbf{Code Listing for Double Pendulum:} Below is the core script,
\texttt{double\_pendulum.py},
which implements the double pendulum with a custom pastel palette.

\begin{lstlisting}[language=Python, caption=Double Pendulum Simulation (double\_pendulum.py), label=lst:doublependulum]
#!/usr/bin/env python3
import math
import numpy as np
from vpython import (
    canvas, vector, color, sphere, cylinder,
    rate, distant_light, local_light, box
)

def hex_to_rgbnorm(hex_str):
    """Convert a hex color string to a normalized RGB vector for VPython."""
    hex_str = hex_str.lstrip('#')
    r = int(hex_str[0:2], 16) / 255.0
    g = int(hex_str[2:4], 16) / 255.0
    b = int(hex_str[4:6], 16) / 255.0
    return vector(r, g, b)

# Pastel color palette
BACKGROUND_PASTEL_HEX = '#FFF0F5'
PIVOT_COLOR_HEX       = '#EE82EE'
ROD1_COLOR_HEX        = '#8F8788'
ROD2_COLOR_HEX        = '#8F8788'
MASS1_COLOR_HEX       = '#9370DB'
MASS2_COLOR_HEX       = '#DA70D6'
MASS2_TRAIL_COLOR_HEX = '#DA70D6'
FLOOR_COLOR_HEX       = '#D3D3D3'

def double_pendulum_derivs(y, params):
    """Derivatives of the planar double pendulum."""
    theta1, omega1, theta2, omega2 = y
    m1, m2, L1, L2, g = params

    sin1 = np.sin(theta1)
    sin2 = np.sin(theta2)
    sin12 = np.sin(theta1 - theta2)
    cos12 = np.cos(theta1 - theta2)

    denom = 2*m1 + m2 - m2 * np.cos(2*theta1 - 2*theta2)

    alpha1 = (
        -g*(2*m1 + m2)*sin1
        - m2*g*np.sin(theta1 - 2*theta2)
        - 2*sin12*m2*(omega2**2*L2 + omega1**2*L1*cos12)
    ) / (L1 * denom)

    alpha2 = (
        2*sin12 * (
            omega1**2*L1*(m1 + m2)
            + g*(m1 + m2)*np.cos(theta1)
            + omega2**2*L2*m2*cos12
        )
    ) / (L2 * denom)

    return np.array([omega1, alpha1, omega2, alpha2], dtype=float)

def rk4_step(y, dt, derivs_func, params):
    """One RK4 integration step."""
    k1 = derivs_func(y, params)
    k2 = derivs_func(y + 0.5*dt*k1, params)
    k3 = derivs_func(y + 0.5*dt*k2, params)
    k4 = derivs_func(y + dt*k3, params)
    return y + (dt/6.0)*(k1 + 2*k2 + 2*k3 + k4)

def run_simulation():
    # Physical parameters
    m1, m2 = 2.0, 1.0
    L1, L2 = 1.5, 1.5
    g = 9.81
    params = (m1, m2, L1, L2, g)

    # Initial conditions
    theta1_0, omega1_0 = 1.2, 0.0
    theta2_0, omega2_0 = 2.1, 0.0
    y = np.array([theta1_0, omega1_0, theta2_0, omega2_0], dtype=float)

    dt = 0.0005
    sim_duration = 40.0
    N_SUBSTEPS = 10
    steps = int(sim_duration / dt / N_SUBSTEPS)

    # 3D scene
    scene = canvas(
        width=1280,
        height=720,
        center=vector(0, 1.0, 0),
        background=hex_to_rgbnorm(BACKGROUND_PASTEL_HEX),
        fov=0.0075
    )

    # Custom lights
    scene.lights = []
    distant_light(direction=vector(1, 1, 1), color=color.white)
    local_light(pos=vector(-2, 3, 2), color=vector(0.7, 0.7, 0.7))
    local_light(pos=vector(2, -3, 3), color=vector(0.5, 0.5, 0.5))

    floor = box(
        pos=vector(0, -2.5, 0),
        size=vector(5, 0.05, 5),
        color=hex_to_rgbnorm(FLOOR_COLOR_HEX),
        opacity=0.2
    )

    pivot = vector(0, 1.5, 0)
    pivot_sphere = sphere(
        pos=pivot,
        radius=0.015,
        color=hex_to_rgbnorm(PIVOT_COLOR_HEX),
        shininess=0.6
    )

    rod1 = cylinder(
        pos=pivot,
        axis=vector(0, 0, 0),
        radius=0.03,
        color=hex_to_rgbnorm(ROD1_COLOR_HEX),
        shininess=0.6
    )
    rod2 = cylinder(
        pos=pivot,
        axis=vector(0, 0, 0),
        radius=0.03,
        color=hex_to_rgbnorm(ROD2_COLOR_HEX),
        shininess=0.6
    )

    mass1 = sphere(
        pos=vector(0, 0, 0),
        radius=0.1*math.sqrt(2),
        color=hex_to_rgbnorm(MASS1_COLOR_HEX),
        shininess=0.6,
        make_trail=False
    )
    mass2 = sphere(
        pos=vector(0, 0, 0),
        radius=0.1,
        color=hex_to_rgbnorm(MASS2_COLOR_HEX),
        shininess=0.6,
        make_trail=True,
        trail_radius=0.01,
        trail_color=hex_to_rgbnorm(MASS2_TRAIL_COLOR_HEX),
        retain=5000
    )

    mass2.clear_trail()

    rod1.length = L1
    rod2.length = L2

    CAPTURE_FRAMES = False
    frame_count = 0

    # Main loop
    for i in range(steps):
        # RK4 substeps
        for _ in range(N_SUBSTEPS):
            y = rk4_step(y, dt, double_pendulum_derivs, params)

        rate(200)

        theta1, omega1, theta2, omega2 = y
        x1 = L1 * np.sin(theta1)
        y1 = -L1 * np.cos(theta1)
        x2 = x1 + L2 * np.sin(theta2)
        y2 = y1 - L2 * np.cos(theta2)

        pos1 = vector(x1, y1, 0) + pivot
        pos2 = vector(x2, y2, 0) + pivot

        rod1.pos = pivot
        rod1.axis = pos1 - pivot
        rod2.pos = pos1
        rod2.axis = pos2 - pos1

        mass1.pos = pos1
        mass2.pos = pos2

        try:
            mass2.trail_object.opacity = 0.5
        except Exception:
            pass

        if CAPTURE_FRAMES:
            scene.capture(f"frame{i:04d}.png")
            frame_count += 1

    print("Simulation complete.")

if __name__ == "__main__":
    run_simulation()
\end{lstlisting}

\textbf{Key Implementation Details:}
\begin{enumerate}[itemsep=2pt, label=(\roman*)]
    \item \textit{Pastel Palette:} All color definitions are in HEX, converted to normalized RGB for VPython.
    \item \textit{Trail Visualization:} Only the second mass (\texttt{mass2}) has a trail to highlight the complexity of its motion.
    \item \textit{Frame Capture:} If \texttt{CAPTURE\_FRAMES} is set to \texttt{True}, each rendered frame is saved as a PNG image for creating animations.
    \item \textit{Performance Tuning:} The \texttt{dt} (time step) and \texttt{N\_SUBSTEPS} can be adjusted for finer or coarser simulation detail.
\end{enumerate}

% ----------------------------------------------------------------------------------
\item \textbf{N-Pendulum Implementation (Generalized)}
\label{sec:NPendulumImplementation}

We also provide a script for simulating an $N$-link pendulum in \texttt{N\_pendulum.py}, which builds the mass matrix $\mathbf{M}$ and the forcing vector $\mathbf{f}$ by directly translating the summation-based formulas (e.g., Equation~(86) in \cite{yesilyurt2020equationsmotionformulationpendulum}). An $N\times N$ linear system is solved at each time step to find the angular accelerations. The code follows an RK4 approach, similarly to the double pendulum script.

Below is a brief excerpt showing the main matrix/forcing construction (for reference only):

\begin{lstlisting}[language=Python, caption=Excerpt from N\_pendulum.py (Matrix Construction)]
def build_M_and_f(angles, omegas, lengths, masses, g):
    N = len(angles)
    Mmat = np.zeros((N, N), dtype=float)
    fvec = np.zeros(N, dtype=float)
    ...
    # Summation-based approach (Equation 86 in Yesilyurt reference)
    # Mmat[j,j], Mmat[j,k], and fvec[j] get contributions from:
    #  g * l_j * sin(theta_j)* m_k, velocity coupling, etc.
    ...
    return (Mmat, fvec)
\end{lstlisting}

Because each time step requires $\mathcal{O}(N^3)$ operations (due to matrix inversion or solving), the simulation may slow down for larger $N$. Nonetheless, it effectively illustrates the complexity and chaotic behavior of higher-order pendulum systems.

% ----------------------------------------------------------------------------------
\item \textbf{Usage and Customization}
\label{sec:usage}

\textbf{Running the Double Pendulum:}
\begin{verbatim}
python double_pendulum.py
\end{verbatim}
A new browser window will open, displaying the real-time 3D animation of the double pendulum.

\textbf{Running the N-Pendulum:}
\begin{verbatim}
python N_pendulum.py
\end{verbatim}
Here, you can modify \texttt{N} (the number of pendulum links) as well as the arrays for rod lengths and masses, and choose different initial angles in the \texttt{run\_simulation()} function.

\textbf{Changing Parameters:} For both scripts, key parameters include masses, rod lengths, gravitational acceleration, and initial angles. These can be modified directly in \texttt{run\_simulation()} to explore different dynamical regimes.

\textbf{Capturing Frames:} If you wish to create a video of either simulation, enable the respective capture flags (e.g., \texttt{CAPTURE\_FRAMES}) if present or incorporate \texttt{scene.capture(...)} within the main loop.

% ----------------------------------------------------------------------------------
\item \textbf{Results and Observations}

Both the double pendulum and the $N$-pendulum exhibit extreme sensitivity to initial conditions. Small changes in angles, rod lengths, or masses can result in dramatically different trajectories. Over longer simulation times, the paths of the various bobs often fill significant portions of the accessible phase space, illustrating chaotic behavior.

% ----------------------------------------------------------------------------------
\item \textbf{Conclusion}

We presented Python-based simulations of double and $N$-link pendulums with a focus on readability, extensibility, and visual appeal. Students and researchers in classical mechanics or chaos theory can modify the code to investigate various phenomena, such as energy transfer, Lyapunov exponents, or the continuum limit as $N\to\infty$.

% ----------------------------------------------------------------------------------
% \item \textbf{References}

\bibliographystyle{ACM-Reference-Format}
\begin{thebibliography}{00}

\bibitem{doublependulum1}
Guckenheimer, J. and Holmes, P. (2013). \textit{Nonlinear Oscillations, Dynamical Systems, and Bifurcations of Vector Fields}. Springer.

\bibitem{doublependulum2}
Strogatz, S. (2018). \textit{Nonlinear Dynamics and Chaos with Applications to Physics, Biology, Chemistry, and Engineering}. CRC Press.

\bibitem{vpython}
Sherwood, B. and others. (2025). \textit{VPython Library: Real-time 3D Graphics for Python}. \url{https://vpython.org}

\bibitem{yesilyurt2020equationsmotionformulationpendulum}
Yesilyurt, B. (2020). \textit{Equations of Motion Formulation of a Pendulum Containing N-point Masses}. arXiv e-prints. \url{https://arxiv.org/abs/1910.12610}

\end{thebibliography}

\end{enumerate}

\thispagestyle{ack_footer}

\end{document}
